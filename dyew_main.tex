\documentclass[11pt]{cernrep}
\usepackage{graphicx,epsfig}
\usepackage{amsmath}
\bibliographystyle{lesHouches}

% definitions here, only use \newcommand, no \def
\newcommand{\order}{\mathcal{O}	}


\begin{document}

% find a catchy title, this is just a filler
\title{Electroweak corrections in Drell-Yan production}

\author{
  A.\ Huss$^1$,
  M.\ Sch\"onherr$^2$
}
\institute{
  $^1$ ETH\\
  $^2$ UZH
}

\maketitle

\begin{abstract}
  We compare the description of higher-order electroweak corrections at 
  $\order(\alpha)$ and $\order(\alpha_S\alpha)$ in Drell-Yan production.
\end{abstract}

\section{INTRODUCTION}
\label{sec:dyew:intro}



\section{COMPUTATION}
\label{sec:dyew:comp}


$\order(\alpha_s\alpha)$ corrections are calculated directly in \cite{}.

Another approach to higher order QED or electroweak corrections is 
presented in the soft-photon resummation of Yennie, Frautschi and 
Suura (YFS) \cite{Yennie:1961ad}. Therein the universal structure 
of real and virtual soft photon emissions is exploited to construct 
and all-orders approximation to the process at hand which can be 
systematically supplemented with process-dependent finite hard 
real and virtual emission corrections. The implementation presented 
in \cite{Schonherr:2008av} focusses on higher-order QED corrections 
to particle decays and is used since as the default mechanism for such 
corrections in \textsc{Sherpa} \cite{Gleisberg:2008ta}, both for 
elementary particle (e.g.\ $W^\pm$, $Z$, $\tau^\pm$) as well as 
hadron decays. 

In the present context of lepton pair production the higher-order 
QED corrections are effected in a factorised approach. The complete 
process $pp\to\ell^+\ell^-$ is calculated at LO or NLO in the strong 
coupling constant keeping all off-shell effects. Then, an intermediate 
resonance $X$ is reconstructed from the lepton pair and assigned its 
invariant mass. Its decay width is then corrected for higher order 
QED corrections, including YFS resummation, to 
\begin{equation}
  \begin{split}\label{eq:dyew:comp:yfs}
    \Gamma
    \,=\;&
      \frac{1}{2m_X}\sum\limits_{n_\gamma=0}^\infty\frac{1}{n_\gamma !}
      \int\mathrm{d}\Phi\;e^{Y(\Omega)}
      \prod\limits_{i=1}^{n_\gamma}\mathrm{d}\Phi_i\,\tilde{S}(k_i)\,\Theta(k_i,\Omega)
      \left[
        \tilde\beta_0^0
        +\tilde\beta_0^1
        +\sum\limits_{i=1}^{n_\gamma}\frac{\tilde\beta_1^1(k_i)}{\tilde{S}(k_i)}
        +\order(\alpha^2)
      \right]\;.
  \end{split}
\end{equation}
Therein, $m_X$ is the mass of the decaying resonance and $\mathrm{d}\Phi$ 
is the phase element of the leading order decay, and $\tilde\beta_0^0$ is 
the leading order decay squared matrix element. The $Y(\Omega)$ then 
is the sum of the eikonal approximations to virtual photon exchange and 
unresolved soft real photon emission, $\Omega$ denoting the region in which 
soft photons are not resolvable. The YFS form factor, $e^{Y(\Omega)}$ then 
resums these leading logarithmic universal corrections to all orders. 
Resolved photons are then described explicitly, emission by emission, 
by the eikonal $\tilde{S}$ depending on the individual photon momentum 
$k_i$. $\mathrm{d}\Phi_i$ is the corresponding phase space element. The 
eikonal approximations used in both the YFS form factor and for 
resolved real emissions can then, order-by-order, be corrected by 
supplementing the corresponding infrared-subtracted squared matrix 
elements $\tilde\beta_i^{i+j}$ of $\order(\alpha^{i+j})$ relative to 
the Born decay and containing $i$ resolved photons. Since all charged 
particles are considered massive in the context of YFS resummation, 
all $\tilde\beta_i$ are free of any infrared singularity. Finally, it is 
interesting to note that in the case of multi-photon emission 
each emitted photon receives the hard emission correction $\tilde\beta_1^1$ 
in the respective one-photon emission projected phase space.

The implementation used here, as we restrict the the $\gamma^*/Z$ propagator 
virtuality to be near the $Z$ mass, always identifies the resonance $X$ with 
the $Z$ boson. The calculation thus contains the $\order(\alpha)$ virtual 
corrections $\tilde\beta_0^1$ and real emission corrections $\tilde\beta_1^1$ 
resulting in an NLO QED accurate description. As NLO weak corrections 
are finite they can in principle be incorporated in the $\tilde\beta_0^1$. 
This is left to a future work.


\section{RESULTS}
\label{sec:dyew:results}

definitely bare $m_{\mu\mu}$ and dressed $m_{ee}$, 
maybe also $p_{\perp,e}$, $p_{\perp,\mu}$, $y_e$, $y_\mu$

\section{CONCLUSIONS}
\label{sec:dyew:conclusions}

final remarks

\section*{ACKNOWLEDGEMENTS}

We thank the organisers.

%=================================
\bibliography{dyew_bib}
\end{document}
